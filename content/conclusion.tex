\chapter{Conclusion}
\label{ch:conclusion}

This thesis has demonstrated a method for synthesizing a physics-based mathematical model for MARGE. Data from ground vibration testing was obtained which was used to improve the accuracy of the finite-element model. The finite element model was then the basis for the mathematical model, and physics-based tuning parameters were applied which increased its accuracy significantly. Gradient-based optimization methods were utilized to adjust the tuning parameters to reconcile the frequency response of the mathematical model with that of experimental wind tunnel data. The final mathematical model is a linear, time-invariant state-space model which can be used in the design of aeroelastic flight control laws.

\section{Future Work}

Future work would be helpful in extending three aspects of this study.

First, the experimental data on which the mathematical modeling was based can be improved upon. With modern optical and laser-based mode-shape measurement systems, authoritative mode shape data can be obtained which can augment or even supercede the finite-element model in modeling structural dynamics. Higher quality sensors (such as accelerometers) which are electronically shielded from outside electrical interference could also eliminate significant sources of uncertainty in the initial model, increasing its accuracy.

The mathematical modeling itself could be improved by applying global optimization methods, since it has been demonstrated that there is potential for improvement from an optimization method which can circumvent local minima. The mathematical modeling could also be improved by modeling more damping mechanisms such as static and dynamic friction at the wing root and load-dependent damping.

Lastly, this mathematical model can be tested in active aeroservoelastic control law design. Future work may include synthesis and testing of gust load alleviation, maneuver load alleviation, and flutter suppression control laws using this model. These control laws can then make viable more lightweight aircraft designs which have increased performance over designs with passive structures.