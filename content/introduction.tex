\chapter{Introduction}
\label{ch:introduction}

%%%%%%%%%%%%%%%%%%%%%%%%%%%%%%%%%%%%%%%%%%%%%%%%%%%%%%%%%%%%%%%%%%%%%
%\section{Motivation} %%%%%%%%%%%%%%%%%%%%%%%%%%%%%%%%%%%%%%%%%%%%%%%
%%%%%%%%%%%%%%%%%%%%%%%%%%%%%%%%%%%%%%%%%%%%%%%%%%%%%%%%%%%%%%%%%%%%%
%
%talk about aeronautics
%
%talk about aerodynamics
%
%talk about structural dynamics
%
%talk about aeroelasticity
%
%talk about control
%
%talk about aeroservoelasticity

%%%%%%%%%%%%%%%%%%%%%%%%%%%%%%%%%%%%%%%%%%%%%%%%%%%%%%%%%%%%%%%%%%%%%
\section{Background}
%%%%%%%%%%%%%%%%%%%%%%%%%%%%%%%%%%%%%%%%%%%%%%%%%%%%%%%%%%%%%%%%%%%%%

Active aeroservoelastic control is a key tool in the pursuit of increased aircraft performance. Applications of active control include ride comfort improvement (\cite{Jones1979}), maneuver load alleviation (\cite{Sensburg1982}), gust load alleviation (\cite{Nissim1976}), and flutter suppression (\cite{Livne2018}). Suchs technologies can result in reduced structural weight, increased passenger comfort, or improved handling qualities, all of which lead to a better performing aircraft.

Determination of the aircraft's mathematical model is  is a prerequisite to designing these aeroservoelastic control laws. The mathematical model must capture the relevant physics of the aircraft in order for the control laws designed around it to function properly when implemented. These mathematical models can be deteremined using analytical modeling, experimental data, or both.

System identification from experimental data can be done via black-box system identification (where the mathematical model's structure is arbitrary) or grey-box system identification (where the mathematical model's structure is pre-determined). Unlike mathematical models from black-box system identification, physics-based grey-box mathmatical models are capable of generalizing to conditions outside of those from which they were derived. Such a generalizable mathematical model is desirable for aeroservoelastic systems because it expands the validity of the model to flight conditions outside of those it was explicitly tested in, significantly reducing the cost of testing in order to cover all relevant flight conditions.

%%%%%%%%%%%%%%%%%%%%%%%%%%%%%%%%%%%%%%%%%%%%%%%%%%%%%%%%%%%%%%%%%%%%%
\section{Contribution}
%%%%%%%%%%%%%%%%%%%%%%%%%%%%%%%%%%%%%%%%%%%%%%%%%%%%%%%%%%%%%%%%%%%%%

This thesis describes of the modeling methods and experimental validation work performed to synthesize one such grey-box aeroservoelastic mathematical model for a wind tunnel model named MARGE \cite{Quenzer2019}. The mathematical model takes the form of a linear, time-invariant state-space equation which is amenable for use in the design of aeroservoelastic flight control laws. This mathematical model is based on both analytical modeling and experimental data. The experiments performed include ground vibration testing and wind tunnel testing. The analytical model is augmented using the experimental data to better match the behavior of the wind tunnel model.

%%%%%%%%%%%%%%%%%%%%%%%%%%%%%%%%%%%%%%%%%%%%%%%%%%%%%%%%%%%%%%%%%%%%%
%\section{Prior Work} %%%%%%%%%%%%%%%%%%%%%%%%%%%%%%%%%%%%%%%%%%%%%%%
%%%%%%%%%%%%%%%%%%%%%%%%%%%%%%%%%%%%%%%%%%%%%%%%%%%%%%%%%%%%%%%%%%%%%
%
%Refer to documents from Eli's review papers.
%
%Talk about MARGE Quenzer 2019.

%%%%%%%%%%%%%%%%%%%%%%%%%%%%%%%%%%%%%%%%%%%%%%%%%%%%%%%%%%%%%%%%%%%%
\section{Organization} %%%%%%%%%%%%%%%%%%%%%%%%%%%%%%%%%%%%%%%%%%%%%
%%%%%%%%%%%%%%%%%%%%%%%%%%%%%%%%%%%%%%%%%%%%%%%%%%%%%%%%%%%%%%%%%%%%

The remainer of this thesis is organized as follows: first, the test article used in this study is described in Chapter \ref{ch:sysDescription}. Then, the modeling methods used to generate the state-space model are formulated in Chapter \ref{ch:sysModeling}. The finite-element model and the GVT done to correct it are covered in Chapters \ref{ch:nastran} and \ref{ch:gvt}, respectively. The wind tunnel testing and model tuning performed based on it are covered in Chapters \ref{ch:windTunnel} and \ref{ch:tuning}, respectively. Finally, Chapter \ref{ch:conclusion} concludes with a discussion of the results and potential follow-on studies.