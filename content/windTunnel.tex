\chapter{Wind Tunnel Testing and Model Tuning}
\label{ch:windTunnel}

This chapter describes the wind-tunnel testing of MARGE and the tuning of the aeroservoelastic model to match the experimental data. It also discusses shortcomings of the data and lessons learned.

%%%%%%%%%%%%%%%%%%%%%%%%%%%%%%%%%%%%%%%%%%%%%%%%%%%%%%%%%%%%%%%%
\section{Data Acquisition and Analysis} %%%%%%%%%%%%%%%%%%%%%%%%
%%%%%%%%%%%%%%%%%%%%%%%%%%%%%%%%%%%%%%%%%%%%%%%%%%%%%%%%%%%%%%%%

MARGE's response was tested at the University of Washington's 3x3 low-speed wind tunnel. It was tested at six flight conditions, $q_D=\{60,100,163,207,281,343\}$ Pa. At each flight condition, the response to each of the four inputs was tested three times. For the gust vanes, a discrete 4-degree doublet gust was generated with a frequency of 1.45 Hz (approximately equivalent to the first wing bending natural frequency). For the ailerons, a 5-degree frequency sweep from 1 Hz to 2 Hz was performed. For the elevator, a 2-degree frequency sweep from 1 Hz to 2 Hz was performed. This frequency band was chosen as it encompassed the first wing bending mode and also stayed within the bandwidth of the actuators.

The exception to the above is at $q_d=343$ Pa. At this dynamic pressure, the aileron sweeps were reduced in magnitude to 3.5 degrees and the elevator sweep was reduced in magnitude to 1 degree. This was done in order to reduce the risk of damage to the model due to violent responses at high speeds. <verify with john>

All of these tests were controlled and recorded using Simulink Real-Time. The testing yielded time-series data of input commands and sensor readings for each test.

\subsection{Data Postprocessing}
The time-domain data was post-processed in a similar way as was done for the GVT data in Section \ref{sec:generateFRF}. Each run's data was truncated to start when the input began and end five seconds after the input ended. The data was re-centered to have zero mean and then the three runs of each test were concatenated.

This single concatenated time-domain signal was then buffered into overlapping Hann windows and transformed using the CZT. The FRFs were then computing using Equation \ref{eq:frf}.

\subsubsection{Accelerometer Data Postprocessing}

bandwidth out [1,2]

bandpass in [0.8,25]

re-center


%%%%%%%%%%%%%%%%%%%%%%%%%%%%%%%%%%%%%%%%%%%%%%%%%%%%%%%%%%%%%%%%
\section{Model Tuning} %%%%%%%%%%%%%%%%%%%%%%%%%%%%%%%%%%%%%%%%%
%%%%%%%%%%%%%%%%%%%%%%%%%%%%%%%%%%%%%%%%%%%%%%%%%%%%%%%%%%%%%%%%